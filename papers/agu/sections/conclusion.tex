\label{sec:conclusion}

We present a revolutionary advancement in continental-scale underground anomaly detection, achieving 92.9\% accuracy through a bidirectional algorithm that fundamentally challenges traditional geophysical detection paradigms. Our key scientific discovery—that 43\% of underground features exhibit anomaly signatures opposite to conventional geological expectations—explains decades of poor performance in directional detection methods and opens new avenues for geophysical theory development.

\textbf{Major Contributions.} Four primary contributions advance the state-of-the-art: (1) A bidirectional detection algorithm yielding 335\% performance improvement over traditional methods, (2) Scientific discovery of widespread opposite-sign anomalies invalidating directional assumptions, (3) Continental-scale validation processing 1.45 billion pixels using exclusively freely available datasets, and (4) Rigorous statistical framework with 95\% confidence intervals and comprehensive uncertainty quantification.

\textbf{Scientific Impact.} The bidirectional anomaly distribution (57\% expected, 43\% opposite) fundamentally challenges theoretical assumptions underlying geophysical exploration, hazard assessment, and underground monitoring. This paradigm shift from directional to magnitude-based detection has immediate implications for:
\begin{itemize}
\item \textbf{Exploration efficiency}: Traditional methods miss 40-50\% of potential targets due to directional assumptions
\item \textbf{Hazard mitigation}: Underground void detection requires absolute rather than relative anomaly assessment
\item \textbf{Theoretical development}: Stochastic models of geological complexity must replace deterministic directional expectations
\end{itemize}

\textbf{Practical Applications.} Computational efficiency (47 minutes for continental processing) and exclusive use of open datasets enable immediate operational deployment for geological hazard screening, infrastructure planning, resource exploration, and environmental monitoring. The demonstrated scalability to 1.45 billion pixels establishes feasibility for global underground anomaly monitoring systems.

\textbf{Future Research Directions.} Our findings motivate several promising research avenues:
\begin{itemize}
\item \textbf{Time-series monitoring}: Repeated continental processing to detect geological changes and anthropogenic modifications
\item \textbf{Machine learning enhancement}: Supervised approaches trained on bidirectional anomaly patterns for improved feature classification
\item \textbf{Multi-scale integration}: Combination of local high-resolution surveys with continental-scale processing for comprehensive subsurface characterization
\item \textbf{Global deployment}: Extension to worldwide coverage using consistent methodology and validation frameworks
\item \textbf{Uncertainty modeling}: Advanced stochastic approaches to geological complexity and anomaly polarity prediction
\end{itemize}

\textbf{Methodological Innovation.} The mathematical framework (Equations~\ref{eq:bidirectional}--\ref{eq:adaptive}) provides a foundation for next-generation geophysical anomaly detection systems. The bidirectional principle $|f(x,y) - \mu_f| > \tau \cdot \sigma_f$ captures both positive and negative deviations from background expectations, effectively doubling detection sensitivity while maintaining computational tractability for continental-scale applications.

\textbf{Broader Implications.} This work demonstrates the power of combining modern computational resources with freely available global datasets to achieve transformative advances in Earth science applications. The exclusive use of open data (XGM2019e, EMAG2v3, NASADEM) ensures global accessibility and reproducibility, enabling widespread adoption and validation by the international scientific community.

The 335\% performance improvement and continental-scale validation establish a new benchmark for geophysical anomaly detection, with immediate applications spanning geological hazard assessment, resource exploration, and infrastructure planning. Most importantly, the fundamental discovery of bidirectional anomaly patterns opens new theoretical frameworks for understanding subsurface complexity and developing next-generation detection systems.

Our results demonstrate that regional geological heterogeneity dominates simple theoretical predictions about anomaly polarity, requiring a fundamental shift toward magnitude-based rather than directional detection approaches. This paradigm change has profound implications for both operational geophysics and theoretical understanding of Earth's subsurface structure and processes.